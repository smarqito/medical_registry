No contexto de um exemplo prático da utilização da temática principal da 
unidade curricular, as \textbf{expressões regulares}, foi desenvolvido pelo grupo 
uma solução que responde a todos os requirementos inferidos pelos docentes.

O programa, no seu estado atual, consegue, com sucesso, apresentar todos os \textbf{indicadores}
propostos no enunciado com um \textit{layout} que, na opinião do grupo, apresenta concisamente
as informações chave.

No projeto atual o grupo utilizou a biblioteca \textit{re}, como é o caso das aulas práticas.
Neste sentido, e com o auxílio da equipa docente, muniu os alunos com as ferramentas 
necessárias para poder manipular e utilizar, com sucesso, o paradigma inerente às expressões regulares.

Adicionalmente, estabeleceu-se como prioridade o desenvolvimento do código mais eficiente possível,
nomeadamente na abstração do \textit{templating}.

Com a flexibilidade de execução, permitindo ao utilizador escolher os indicadores a inserir, o 
grupo pretendeu integrar um nível de manuseamento extra, bem como demonstrar a evolução das 
suas capacidades ao nível de programação na linguagem em \textit{Python}.

Concluindo, traduz-se num trabalho no qual o grupo está confiante de que traz todos os elementos 
pretendidos pela equipa docente, tendo ainda ambição na inserção de novos elementos para enriquecer
o projeto final.
