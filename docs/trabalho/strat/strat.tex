No contexto do problema em causa, será necessário abordar dois pontos principais:
\begin{itemize}
    \item Extração e armazenamento da informação do ficheiro \textit{CSV}.
    \item Criação de um \textit{website} com os indicadores necessários.
\end{itemize}

Para extrair a informação do ficheiro \textit{CSV}, como especificado em \ref{subsubsec:padrao}, recorreu-se ao uso de \textbf{expressões regulares} para retirar
de uma linha os parâmetros necessários.
Usando este método para todo o ficheiro, é possível obter, de forma rápida e metódica, os registos de todos os atletas 
do \textit{dataset}.
Com a informação necessária a cada indicador agregada num \textit{dicionário} próprio, será possível manipular 
livremente todos os parâmetros a serem usados para construir a página principal e as respetivas \textit{sub-páginas}.

No âmbito de modularizar o projeto, e construir uma \textbf{arquitetura} mais organizada, o trabalho associado à criação das páginas \textit{HTMl} encontra-se 
particionado por vários ficheiros. 
Neste sentido, cada \textbf{indicador estatístico} corresponderá ao seu \textbf{módulo}, um programa em \textit{Python} responsável por gerá-lo, sendo uma análise mais aprofundada
sobre este tópico localizada em \ref{}.

Não obstante os múltíplos programas criados para popular a página principal, todo este trabalho irá localizar-se no ficheiro \textit{reader.py}. 
Este irá simultaneamente agregar toda a informação a ser usada para cada indicador e usá-la so chamar os módulos criados, culminando na criação do 
\textit{index.html} (\ref{}). 

\dots todos os módulos terão por base um ou mais \textit{templates HTML}, sendo que cada indicador tem a sua própria estrutura de apresentação (\ref{subsubsec:templates}).

A solução tenta abstrair a construção dos diversos indicadores, através de \textit{templating} (\ref{subsubsec:templates}).
Isto é, irá construí-los metodicamente a partir dos \textit{templates} criados, usando a mesma estratégia para todos.
Deste modo, permite-se a adição de novos indicadores sem complicações de construção e torna-se o código mais \textit{clean} e \textit{readable}.

No final, irá ser criado um \textit{website} cuja página principal engloba \textbf{hiperligações} para todas as páginas dos indicadores respetivos.