De maneira a conseguir capturar todas os valores existentes em cada linha do ficheiro
do ficheiro csv, foi analisado o padrão existente e a partir deste especificada uma expressão regular 
capaz de os agrupar.

\begin{minted}[breaklines]{bash}
    (?P<id>\w+),(?P<index>\d+),(?P<date>\d{4}-\d{2}-\d{2}),(?P<primeiro>\w+),
    (?P<ultimo>\w+),(?P<idade>\d+),(?P<genero>[MF]),(?P<morada>\w+),(?P<modalidade>\w+),
    (?P<clube>\w+),(?P<email>.*?),(?P<federado>[true|false]),(?P<resultado>[true|false])
\end{minted}

Na construção da expressão regular foram utilizados os parênteses e a flag $?P<>$
para aceder aos valores de uma forma mais simples e estruturada.