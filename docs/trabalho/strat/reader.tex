O \textit{Reader} é responsável pela leitura linha à linha do ficheiro
csv e pelo armazenamento de todos os dados inerentes ao indicadores estatísticos
gerados.

A cada linha lida do ficheiro são atualizadas variáveis que foram criadas com o intuito de
armazenar as informações relativas a cada indicador:
\begin{itemize}
    \item É criada uma página por cada atleta onde é possível consultar todos as 
    informações relativas a este
    \item Verifica-se se a data lida é uma data extrema, em caso positivo, é adicionado
    o indice do atleta a dicionário do tipo {data mínima : [ids], data máxima : [ids]}
    \item Com base no ano e no género lido é adicionado o indice do atleta
    a um dicionário do tipo {ano : {Masculino : [ids], Feminino : [ids]}}
    \item Com base na idade e no género lido é adicionado o indice do atleta
    a um dicionário do tipo {Mais ou igual a 35 anos : {Masculino : [ids], 
    Feminino : [ids]}, Menos que 35 anos : {Masculino : [ids], 
    Feminino : [ids]}}
    \item Com base na morada lida é adicionado o indice do atleta
    a um dicionário do tipo {morada : [ids]}
    \item Com base no ano e no resultado lido é adicionado o indice do atleta
    a um dicionário do tipo {ano : {Apto : [ids], Não Apto : [ids]}}
    \item Com base no ano e no estatuto de federado lido é adicionado o indice do atleta
    a um dicionário do tipo {ano : {Federado : [ids], Não Federado : [ids]}}
    \item Com base no ano e na modalidade lido é adicionado o indice do atleta
    a um dicionário do tipo {ano : {modalidade : [ids]}} e a modalidade a um 
    \textit{set} de modalidades
\end{itemize}

As variáveis criadas no decorrer da leitura são mais tarde percorridas na geração dos
indicadores, assunto que será abordado em \ref{modulos}.

A lista de ids são guardadas para ser possível apresentar as informações que permitiu
criar aquele indicador, nomeadamente uma lista ordenada alfabeticamente pelos nomes dos 
atletas onde é permitido consultar a página individual de cada um.
