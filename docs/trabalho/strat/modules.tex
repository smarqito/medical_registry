Foi criado um \textit{module} por cada indicador estatístico para que estes fossem
responsáveis pelas variáveis criadas no reader(\ref{reader}) e também pela criação
do \textit{html} correspondente.

A geração do ficheiro \textit{html} passa por percorrer todas os conjuntos de chave-valor
dos dicionários, cada parâmetro do indicador é calculado com base numa lista de ids, que posteriormente é  
transformada numa página \textit{html} com a lista de nomes dos atletas, em que cada um tem uma hiperligação 
para a sua página pessoal de informações. 

Todos os parâmetro calculados e respetivas informações que o permitiram calcular, especificamente o 
\textit{path} para a página dos nomes dos atletas, são armazenados e passados ao módulo template.

Módulo template ... 

Classe jogador ...

O módulo globals é o responsável pela geração das pastas onde é colocado o \textit{output} do pograma
consoante os argumentos de inicalização recebidos. A pasta de \textit{output} gera tantas pastas como 
o número de indicadores a gerar, em cada uma são guardadas as página de nomes referentes aos parâmetros
dos indicadores calculados.

