Foi criado um \textit{module} por cada indicador estatístico para que estes fossem
responsáveis pelas variáveis criadas no reader(\ref{reader}) e também pela criação
do \textit{html} correspondente.

A geração do ficheiro \textit{html} passa por percorrer todas os conjuntos de chave-valor
dos dicionários, cada parâmetro do indicador é calculado com base numa lista de ids, que posteriormente é  
transformada numa página \textit{html} com a lista de nomes dos atletas, em que cada um tem uma hiperligação 
para a sua página pessoal de informações. 

Todos os parâmetro calculados e respetivas informações que o permitiram calcular, especificamente o 
\textit{path} para a página dos nomes dos atletas, são armazenados e passados ao módulo template.

\textbf{Módulo} \textit{Template}

Por forma a tornar a aplicação mais escalável e fácil de manter, optou-se por criar um módulo de \textit{templating}.
Apesar de existir várias ferramentas para o efeito, atendendo ao cariz de aprendizagem inerente ao contexto de desenvolvimento
do presente projeto, optou-se por desenvolver este pequeno módulo de raíz.
Para tal, recorreu-se ao módulo \textit{re} do \textit{Pyhon}.

A estratégia desenvolvida foi orientada ao template,
i.e. apenas será apresentado informação para a qual exista uma variável alocada para a mesma no template.
Assume-se objetos do tipo $list$ ou $dict$. Permite, ainda, a utilização de subtemplates.

Assim, foram utilizadas duas \textit{labels}:
\begin{itemize}
    \item[{{chave}}]{Através da chave é acedido ao valor da variável.}
    \item[{{template, chave}}]{Vai aplicar a template (outro ficheiro html) à chave do dicionário\\
    Pode, ainda, ser utilizado sobre listas. Nesse caso, vai aplicar a template N vezes no valor da chave.}
\end{itemize}

A partir das \textit{labels} definidas, começa-se por iterar os templates procurando por cada tipo (chave ou template).
De forma recursiva, vai-se entrando na estrutura até se chegar à chave que alberga o valor, tratando-se de efetuar \textbf{sub}
da label por este.

Conforme referido, esta estratégia é implementada de forma recursiva, pelo que se consegue aninhamento sem limite de níveis.
Apesar de se ter conseguido um bom resultado, este módulo tem algumas limitações. Estas, poderiam ser ultrapassadas através da utilização de contextos.


A classe \textbf{Jogador} armazena todos os parâmetros lidos numa linha do ficheiro csv. Foi ainda criado um dicionário
que armazena todos os jogadores em que a chave é o id, que é utilizado para gerar as referências na página \textit{html} com 
a lista de nomes dos atletas

O módulo globals é o responsável pela geração das pastas onde é colocado o \textit{output} do pograma
consoante os argumentos de inicalização recebidos. A pasta de \textit{output} gera tantas pastas como 
o número de indicadores a gerar, em cada uma são guardadas as página de nomes referentes aos parâmetros
dos indicadores calculados.

