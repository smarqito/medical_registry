\subsubsection{Argumentos de inicialização}

Quando se inicializa o programa é possível passar argumentos que indicam o tipo de 
indicadores a serem criados e também a pasta onde será guardado o \textit{output}:
\begin{itemize}
    \item [-o]{Pasta de output}
    \item [-d]{Datas extremas}
    \item [-g]{Distribuição por género em cada ano}
    \item [-m]{Distribuição por modalidade em cada ano}
    \item [-i]{Distribuição por idade e género}
    \item [-l]{Distruibuição por morada} 
    \item [-f]{Distribuição por estatuto de federado em cada ano}
    \item [-r]{Percentagem de aptos e não aptos por ano}
\end{itemize}

Caso não sejam passados nenhuns argumentos referentes aos indicadores são
gerados todos por defeito. A pasta de \textit{output} é a \textit{www} quando não
especificada.

Contudo, para poder fornecer ao utilizador todas as ferramentas para a execução do programa sem conhecimento prévio, 
foi criado um \textbf{manual}, escrito pelo grupo, para informar de todas as funcionalidades da aplicação.
Com o comando:
\begin{minted}{bash}
    python3 emdtohtml.py --help
\end{minted}

\dots será apresentado ao utilizador o manual de ajuda.

\inputminted{bash}{manual.txt}

\subsubsection{Redirecionamento do STDIN}

O aplicativo permite utilizar o \textit{stdin} para efetuar a leitura das linhas a serem analisadas.
Assim, caso o utilizador não indique um ficheiro de leitura, o programa ficará a aguardar pela entrada no \textbf{stdin}.
Esta estratégia permite, no entanto, a utilização de pipelines.

\subsubsection{Resultados obtidos}

O \textit{dataset} \textbf{emd.csv} disponiblizado pelos professores na \textit{Blackboard}
foi utilizado como ficheiro de teste. A página \textbf{index.html} criada com todos os indicadores
estatísticos encontra-se nos anexos(ver \ref{anexos}).

