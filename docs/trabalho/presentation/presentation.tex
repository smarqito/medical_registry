\subsubsection{Argumentos de inicialização}

Quando se inicializa o pograma é possível passar argumentos que indicam o tipo de 
indicadores a serem criados e também a pasta onde será guardado o \textit{output}:
\begin{itemize}
    \item [-o]{Pasta de output}
    \item [-d]{Datas extremas}
    \item [-g]{Distribuição por género em cada ano}
    \item [-m]{Distribuição por modalidade em cada ano}
    \item [-i]{Distribuição por idade e género}
    \item [-l]{Distruibuição por morada} 
    \item [-f]{Distribuição por estatuto de federado em cada ano}
    \item [-r]{Percentagem de aptos e não aptos por ano}
\end{itemize}

Caso não sejam passados nenhuns argumentos referentes aos indicadores são
gerados todos por defeito. A pasta de \textit{output} é a www quando não
especificada.

\subsubsection{Resultados obtidos}

O \textit{dataset} \textbf{emd.csv} disponiblizado pelos professores na \textit{Blackboard}
foi utilizado como ficheiro de teste. A página \textbf{index.html} criada com todos os indicadores
estatísticos foi a seguinte: